In this paper results are presented that compare the performance of semi-orthogonal user selection (SUS) scheme assuming MRT beamforming in the MU-MIMO downlink to a brute-force approach to user selection. The brute-force approach represents a computation-intensive exhaustive search for the best performance associated with a group of users chosen from a larger group of candidate users. In this way, the exhaustive search for the best possible performance is a best-case benchmark to compare against the SUS scheme, which reduces the size of the search space, and thus the computational complexity.

The focus of this paper is to compare the performance of the SUS, constrained case, to the exhaustive, unconstrained case; computational complexity will be addressed separately in future work. The performance metric of interest is the sum rate of a group of four users chosen from a larger set of candidate users. The findings presented here suggest that the SUS scheme has a low probability of meeting the exhaustive case. However, as the number of candidate users becomes larger, the SUS constraints do better at filtering the search space. Put differently: as the number of candidate users increases, the probability that the group associated with the best sum rate in the exhaustive search space overlaps with the much smaller SUS-constrained search space. For example, the probability that the SUS search space contains the group with the best sum rate for 30 candidate users is approximately $2\times10^{-4}$. However, as the number of candidate users is increased to 1000, the probability of the such an overlap is increased to 0.13. Thus, as the number of candidate users becomes large, as in the case of a dense wireless deployment, user selection based on channel orthogonality and gain becomes increasingly attractive from a performance point of view. Moreover, the results presented here suggest that lower bound on the expected sum rate approaches the expected maximum sum rate as the number of candidate users becomes large.

The results presented in this paper are conservative. The sum rate results associated with SUS-constrained cases are characterized by lower bounds, while the unconstrained results are not subjected to such lower bounds. 

SUS scheme does not achieve performance as high as the exhaustive search. However, this is neither surprising, nor the argument presented here. The purpose of this effort is to quantify the losses in sum rate performance. Such quantitative analysis is necessary in order to argue a reasonable performance-complexity trade-off.